In this section the results of the project is summarized for some sequences chosen together with the other group. The sequences, as well as the ground truth used are provided by the EC Funded CAVIAR project, \cite{CAVIAR}.

\subsection{Evaluation Scores}

\begin{table}
\centering
	\begin{tabular}{r | c | c | c }
		\emph{Sequence Name}		& \emph{MOTA} & \emph{MOTP} \\
		\hline \hline
		OneLeaveShop1front			& 0.56 & 5.01 \\
		OneLeaveShop2front			& 0.70 & 5.30 \\
		OneLeaveShopReenter2front	& 0.65 & 5.09 \\
		OneStopNoEnter2front 		& 0.85 & 5.48 \\
		WalkByShop1front 			& -0.66 & 7.96 \\
	\end{tabular}
	\caption{\textit{Tracking performance according to the MOTA and MOTP evaluation standards, as described in section \ref{sec:evaluation}.}}
	\label{tab:evaluation_performance}
\end{table}

\subsection{Parameters}

\begin{table}
\centering
	\begin{tabular}{r | c || c | c || c | c | c | c | c }
	&	\multicolumn{1}{|c||}{BG model} & \multicolumn{2}{c||}{FG Segmentation} & \multicolumn{4}{c|}{Shadow Detection} \\
		\hline
		\emph{Sequence Name} & \emph{Learning Rate} & \emph{Iterations} & \emph{Min Thickness} &\emph{$\tau_H$} & \emph{$\tau_S$} & \emph{$\alpha$} & \emph{$\beta$}\\ 
		\hline \hline
		OneLeaveShop1front			& 0.05 		& 3 & 3.5 	& 0.5 & 1 & 0.8 & 0.99\\
		OneLeaveShop2front			& 0.05 		& 3 & 4 	& 0.5 & 1 & 0.8 & 0.99\\
		OneLeaveShopReenter2front	& 0.05		& 4 & 4 	& 0.5 & 1 & 0.8 & 0.99\\
		OneStopNoEnter2front 		& 0.033		& 3 & 4 	& 0.5 & 1 & 0.8 & 0.99\\
		WalkByShop1front 			& 0.033 	& 4 & 8 	& 0.5 & 1 & 0.8 & 0.99\\
	\end{tabular}
	\caption{\textit{Parameters used to receive the evaluation results presented in table \ref{tab:evaluation_performance}.}}
	\label{tab:evaluation_parameters}
\end{table}


\subsection{Performance}


\subsection{Technical Conclusion}
The evaluation needs access to some sort of ground truth which is defined as the best possible achievable tracking output. Through out this part of the document an object is defined as a position by the ground truth and a hypothesis as the output from the tracker. The method only allows one-to-one correspondence between objects and hypothesis and in case of conflict the combination yielding the lowest total distance error is chosen.

\subsection{Conclusion}
MOTA is measure of accuracy with respect to how many mistakes are made by the tracker. It consists of four variables: misses, false positive, mismatches and number of objects. A miss is when no hypothesis is suggested close enough to an object. Close enough is defined by a threshold, T. This is the only design parameter in this method. If the distance between an object and the closest hypothesis is larger than T, the object yields a miss. If a hypothesis has no object within the threshold it results in a false positive error. One important feature of a tracker is the ability to keep objects identities correct. If this is not the case and an object is changing identity between frames one mismatch error is added for every change. The number of objects variable is defined as the total number of objects trackable according to the ground truth in the current frame. For a sequence the equation for MOTA is found in \eqref{eq:MOTA}.




