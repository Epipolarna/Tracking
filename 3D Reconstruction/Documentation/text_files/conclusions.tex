
\subsection{Point Extractor}


\subsubsection{Possible improvements}


\subsection{Pose Estimator}


\subsubsection{Possible improvements}


\subsection{Non-linear Optimizer}


\subsubsection{Possible improvements}


\pagebreak
\subsection{3D Visualization}
Implementing this part of the system in opengl was perhaps not the wisest decision, as the ammount of work and difficilty to debug opengl calls made progress rather slow.

It did however make it easy to implement project-specific functions into the visualizer without having to learn yet another API.

\subsubsection{Possible improvements}
It should be possible to color the points according to the image data they are extracted from. If more points where obtained it should be possible to draw them only as colored dots in space.

Further one might create a mesh between the detected points, to generate a full 3D-Model. It should then also be possible to texture it using data from the images to create a complete 3D representation of the original physical object.