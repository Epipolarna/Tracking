There are two cases of finding the pose of a camera. One is when there are only two cameras available, and the other one when there are several cameras already orientated to each other and one camera is being added to that group. In the first case an algorithm described in \cite{Klas} is being used. In the second a PnP (Perspective n-Point) algorithm is used.

\subsubsection{Alg. 5}
Given the fundamental matrix, $F$, for an image pair and the internal camera parameters, $K$, for the camera responsible for images, it is possible to estimate the two camera poses relative one another up to an unknown scale factor. This can be done in different ways, but since the camera is calibrated and the same for all images it is convenient to use algorithm 5 in \cite{Klas}. This uses a singular value decomposition of the essential matrix, \eqref{eq:EssentialMatrix} to find the translation, $t$, and rotation, $\boldsymbol{R}$, of camera two relative camera one. A minor drawback with the algorithm is that it produces four different combinations of rotations and translations the al need to be tested to find the correct one. For a more in detail description of the mathematical motivations of this algorithm see \cite{Klas}. 

\begin{equation}
\label{eq:EssentialMatrix}
\boldsymbol{E} = \boldsymbol{K}^T \boldsymbol{FK}
\end{equation}

\begin{equation}
\boldsymbol{U S V}^T = \boldsymbol{E}
\end{equation}

\begin{equation}
t_1 = v3, \hspace{1 cm}  \boldsymbol{V} = \begin{pmatrix}
										 |  & |  & |  \\
										 v1 & v2 & v3 \\
										 |  & |  & |
										\end{pmatrix}
\end{equation}

\begin{equation}
t_2 = -t_1
\end{equation}

This translation has an unknown scaling yielding the possibility of both plus and minus in sign. To find the rotation an auxiliary matrix, $\boldsymbol{W}$, is defined. Also to avoid mirroring in the rotation, the suggestions are multiplied with corresponding determinant.

\begin{equation}
\boldsymbol{W} = 	\begin{pmatrix}
					0  & 1 & 0 \\
					-1 & 0 & 0 \\
					0  & 0 & 1
					\end{pmatrix}
\end{equation}

\begin{equation}
\hat{\boldsymbol{R}}_1 = \boldsymbol{U W }^{-1} \boldsymbol{V}^T
\end{equation}

\begin{equation}
\hat{\boldsymbol{R}}_2 = \boldsymbol{U W }^{-T} \boldsymbol{V}^T
\end{equation}

\begin{equation}
\boldsymbol{R}_1 = det(\hat{\boldsymbol{R}}_1) \hat{\boldsymbol{R}}_1 
\hspace{2 cm}
\boldsymbol{R}_2 = det(\hat{\boldsymbol{R}}_2) \hat{\boldsymbol{R}}_2
\end{equation}

\begin{equation}
\boldsymbol{C}_{ij} = (\boldsymbol{R}_i | t_j)
\end{equation}

To find out which combination to chose, one corresponding point is chosen from the image pair. This point is then triangulated for the four camera pairs. Three of these triangulations should end up with the triangulated 3D behind the camera leaving only one camera left which is the one we choose. 

\subsubsection{PnP}
The PnP algorithm is used to fit an external camera to an existing set of cameras with corresponding triangulated 3D points. The algorithm tries to adjust the new camera's rotation and translation to minimize the re-projection error of the existing 3D points that by 2D correspondence are known to exist in the new camera as well. In the this procedure there is an optional outlier removal part where points producing too large re-projection errors are discarded from future 3D reconstruction. 