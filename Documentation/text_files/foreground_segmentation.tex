The Foreground segmentation is performed by the foreground processor module. The purpose of the foreground processor is to decide which regions are interesting, based on the probability map, and to create an object for each interesting region, and add these to the object vector of the frame. The shadow suppression is a part of the foreground processing. \\
\newline
The foreground processor is initiated by setting appropriate parameter values using the \emph{init} command. Which parameters that are suitable depends entirely on the image sequence being analyzed. After being initiated, the foreground processor is called using the \emph{segmentForeground} command, which takes the current frame as an input. For more info, see appendix \ref{sec:ForeGroundSeg_code}.

\subsubsection{Morphological cleanup}
The first step in the foreground segmentation is to call the shadow suppression algorithm to get rid of shadows, which the background model can't handle (see section 2.5 of this document). The output of the shadow suppression algorithm is then subject to a distance filter. The distance filter extracts all the outer contours of all the regions with non-zero pixels. Each pixel in the contour has its distance to the edge measured. If the maximum pixel distance does not exceed a certain threshold, the contour is assumed to be garbage and all pixel values are set to zero. Finally a dilate operation is performed to make sure that the remaining regions that are close to each other "melt" together. For example, we do not want the legs and body of a person to be classified as two different objects just because the belt of that person had the same color as the background. 

%\begin{figure}[htb]
%	\centering
%	\includegraphics[width=\linewidth]{images/acatisfinetoo}
%	\caption{\textit{Some nice figure perhaps.}}
%	\label{fig:foreground_segmentation_fig} %Skapar referens till figuren
%\end{figure}

\subsubsection{Object detection}
To detect the regions the OpenCV command \emph{findContours} is used. For each detected contour, if it is large enough (to remove more noise), a bounding rectangle of the type \emph{cv::Rect} is created and, from this rectangle an \emph{object} is created. All of the detected objects are then put into the Frame's object vector, and the foreground processing is complete.

