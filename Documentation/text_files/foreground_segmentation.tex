The Foreground segmentation is performed by the foreground processor module. The purpose of the foreground processor is to, from the probability map, decide which regions are interesting and create an object for each interesting region, and add these to the object vector of the frame. The shadow suppression is a part of the foreground processing. \\
\newline
The foreground processor is initiated by setting appropriate parameter values using the \emph{init} command. After being initiated, the foreground processor is called using the \emph{segmentForeground} command, which takes the current frame as an input. For more info, see appendix \ref{sec:ForeGroundSeg_code}.

\subsubsection{Morphological cleanup}
The first step in the foreground segmentation is to perform an initial simple cleanup of the background using one erode iteration. This performs a very good cleanup since most of the noise in the backround model is in form of single pixels, all of which are removed by the erode operation. After the first erode the shadow suppression algorithm is called to get rid of shadows, which the background model can't handle (see section 2.5). The output of the shadow suppression algorithm is then subject to several more steps of erode/dilate operations before the actual object detection.

%\begin{figure}[htb]
%	\centering
%	\includegraphics[width=\linewidth]{images/acatisfinetoo}
%	\caption{\textit{Some nice figure perhaps.}}
%	\label{fig:foreground_segmentation_fig} %Skapar referens till figuren
%\end{figure}

\subsubsection{Object detection}
To detect the regions the OpenCV command \emph{findContours} is used. For each detected contour, if it is large enough (to remove more garbage), a bounding rectangle of the type \emph{cv::Rect} is created and, from this rectangle an \emph{object} is created. All of the detected objects are then put into the Frame's object vector, and the foreground processing is complete.